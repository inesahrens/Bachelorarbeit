\chapter{Modellierung}
\label{chap:modellierung}


Erinnern wir uns an das Beispiel am Anfang: Ein Blatt Papier befindet sich in Ruheposition auf dem Gebiet $\Omega \subset \R^3 $. Nun wird es an zwei gegen�berliegenden Seiten eingespannt und an diesen Seiten wird gezogen. Zun�chst deformiert sich das Papier. Folgende Modellierung stammt aus \cite{braess}. Die Funktion $\varphi: \Omega \rightarrow \R^3$ beschreibt die neue Position des Papiers. Also gilt $\varphi=u+id$, wobei $u: \Omega \rightarrow \R^3$ die Verschiebung des K�rpers ist. 

Das Ziel ist es, ein $\varphi$ zu finden, bei dem die Verzerrung des Papiers minimal ist. Dazu betrachten wir den Abstand zwischen $x \in \Omega$ und $ x+h \in \Omega$ im aktuellen Zustand. Durch taylorn erhalten wir
\begin{align*}
	\|\varphi(x+h)- \varphi(x)\|^2 = \|\nabla \varphi \|^2 + o(\|h\|^2) = h^T \nabla \varphi^T \nabla \varphi h + o(\|h\|^2).
\end{align*} 
Damit haben wir die Matrix $C:=\nabla \varphi^T \nabla \varphi $ erhalten, die die lokale �nderung der L�ngen angibt und Cauchy-Grennscher Verzerrungstensor genannt wird. 
Die Verzerrung $E$ l�sst sich �ber $ \frac{1}{2} (C-I)$ berechnen. Da $u=\varphi - \text{id}$ gilt, k�nnen wir die Ableitung von u einsetzen und erhalten
\begin{align*}
	E_{ij}=\frac{1}{2} \left( \frac{\partial  u_i}{\partial x_j} +\frac{\partial  u_j}{\partial x_i}  \right) + \frac{1}{2} \sum\limits_{k} \frac{\partial  u_i}{\partial x_k} \frac{\partial  u_j}{\partial x_k} .
\end{align*}
Da im Linearen die quadratischen Terme vernachl�ssigt werden k�nnen, erhalten wir 
\begin{align*}
	\frac{1}{2} \left( \frac{\partial  u_i}{\partial x_j} +\frac{\partial  u_j}{\partial x_i} \right) = \frac{\nabla u^T + \nabla u}{2}
\end{align*}

Irgendwann h�lt das Papier den auf es wirkenden Kr�ften nicht mehr stand und rei�t. Sobald dies geschieht, gilt die Formel f�r die Verzerrung nicht mehr. Also brauchen wir einen Term, der angibt, wo und wie stark der K�rper gerissen ist. Dazu definieren wir $v: \Omega \rightarrow \R$. $v$ ist 0, falls der K�rper vollst�ndig gerissen ist und 1, wenn kein Riss vorhanden ist.  Multiplizieren wir also $ \frac{1}{2} (\nabla u^T + \nabla u)$ mit  $v^2 + \epsilon_1$. Es ergibt sich 
\begin{align*}
	& \min\limits_u \int\limits_{\Omega} |\nabla u|^2 (v^2 + \epsilon_1) \\
	& u=u_0 \text{ auf } \Gamma_1 \cup \Gamma_2  
\end{align*}
Da wir $u$ nur im zweidimensionalen betrachten, k�nnen wir  $\frac{\nabla u + \nabla u^T}{2}^2$  zu $|\nabla u|^2$ vereinfachen. Am rechten bzw. linken Rand $\Gamma_1, \Gamma_2 \subset \Omega$ ist das Material befestigt, was $u=u_0$ aussagt.  

Ideal w�re es, wenn der K�rper niemals rei�en w�rde. Da dies aber in der Anwendung nicht der Fall sein kann, m�chten wir einen m�glichst kleinen Riss erhalten. Die Oberfl�che, auf die der Riss auftritt sollte also minimal sein. Wir wollen also, dass $\int\limits_{\Omega} (1-v)^2$ minimal ist. Aber in der Realit�t stellt man fest, dass der Riss glatt ist. Um dies zu modellieren, addieren wir die Ableitung zum Quadrat mit $\epsilon_2^2$ multipliziert hinzu. Daraus ergibt sich eine Darstellung vom Riss nach \cite{braides}
\begin{align*}
	\epsilon_2 |\nabla v|^2 + \frac{1}{\epsilon_2} (1-v)^2
\end{align*}
Zusammengefasst ist das gesamte Problem gegeben durch
\begin{align*}
	& \min\limits_{u \in  \h^2 , v \in \h } \int_{\Omega} \left( v^2 + \epsilon_1 \right) | \nabla u|^2 + \nu \left(\epsilon_2 |\nabla v|^2 + \frac{1}{\epsilon_2} \left( 1- v \right)^2 \right) dx \\
	& \text{s.d.} \hspace{1ex} 0 \le v \le v_0 \\
	& u=u_0 \text{ auf } \Gamma_1 \cup \Gamma_2 , 
\end{align*}
dabei ist $\nu$ eine Konstante, $\epsilon_i>0$ f�r alle $i \in \{1,2\}$ ein kleiner Parameter und $\Gamma_1, \Gamma_2 \subset \Omega$  der rechte bzw. linke Rand eines rechteckigen Gebietes $\Omega \subset \R^2$, d.h., 
\begin{align*}
	\Omega=[0,a] \times [0,b] \hspace{1ex} \Gamma_1=\{0\} \times [0,b] \hspace{1ex} \Gamma_2=\{a\} \times [0,b] .
\end{align*}
