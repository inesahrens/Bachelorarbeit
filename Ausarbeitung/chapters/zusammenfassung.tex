% ===========================
% Zusammenfassung / Abstract
% ===========================

% --- Erzeugt eine formatierte unnummerierte Ueberschrift 'Zusammenfassung' ohne Eintraege in die .toc-Datei oder fuer die Kolumnenbeschriftung
\chapter*{Zusammenfassung}

Wie der Name schon sagt, soll der Abstract den Inhalt der Arbeit zusammenfassen. Es sollte ausreichend viele Details enthalten, sodass der Leser die M�glichkeit hat zu entscheiden, ob er die vollst�ndige Arbeit lesen soll. In der Regel gilt: Je k�rzer, desto besser, unter der Bedingung, dass es ausreichend viele Informationen enth�lt. Im Allgemeinen ist es empfehlenswert den Abstract zum Schluss der Arbeit zu verfassen, insbesondere nachdem der Hauptteil der Arbeit aufgeschrieben ist. Und noch einige besondere Hinweise f�r die Erstellung eines Abstracts.

\begin{itemize}
	\item Idealerweise sollte der Leser keine Informationen in der Arbeit nachschlagen, um den Abstract zu verstehen.
	\item Vermeiden Sie wenn m�glich mathematische Gleichungen im Abstract.
	\item Versuchen Sie den Abstract leicht verst�ndlich zu gestalten.
\end{itemize}