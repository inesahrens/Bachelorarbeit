% ====================================
% Einleitung
% ====================================

\chapter{Einleitung}
\label{chap:einleitung}

%todo einleitung besser schreiben.
% dazu geh�rt:
% - Anwendung/ Erkl�rungen
%   - Entstehung von Rissen und ihre Fortsetzung in einem Phasenfeld
%   - einzelne Terme erkl�ren
%   -  
% - Organisation der Arbeit am ende der Einleitung

Ich befasse mich mit Rissen und der Darstellung des Problems als Newtonmethode. Dabei sind meine Aufgaben:

\begin{enumerate}
	\item Optimalit�tsbedingungen aufstellen (KKT)
	\item Das Problem auf Semidifferenzierbarkeit untersuchen 
	\item Newtonmethode f�r das Problem aufstellen
	\item Problem implementieren mithilfe von Finite Elemente
	\item Konvergenz des Problems anhand der Implementation untersuchen: H�ngt die Konvergenz vom Gitter ab?
\end{enumerate} 

Das Problem lautet:
\begin{align*}
	& \min\limits_{u \in  \h^2 , v \in \h } \int_{\Omega} \left( v^2 + \epsilon_1 \right) | \nabla u|^2 + \epsilon_2 |\nabla v|^2 + \frac{1}{\epsilon_3} \left( 1- v \right)^2 dx \\
	& \text{s.d.} \hspace{1ex} 0 \le v \le v_0 \\\
	& u=u_0 \text{ auf } \Gamma_1 \cup \Gamma_2  
\end{align*}

\begin{nota}
 $| \nabla u|^2:=\frac{\partial u_1}{\partial x_1 }^2 
 				+ \frac{\partial u_1}{\partial x_2 }^2 
 				+ \frac{\partial u_2}{\partial x_1 }^2 
 				+ \frac{\partial u_2}{\partial x_2 }^2  
 				= u_{1 x_1}^2+ u_{1 x_2}^2 + u_{2 x_1}^2 + u_{2 x_2}^2  $ 
\end{nota} 

dabei ist $\epsilon_i>0 \hspace{1ex}\forall i \in \{1,2,3\}$ ein kleiner Parameter und $\Gamma_1, \Gamma_2 \subset \Omega$  ein der rechte bzw. linke Rand eines rechteckigen Gebietes $\Omega \subset \R^2$, das hei�t, $\Omega=[0,a] \times [0,b] \hspace{1ex} \Gamma_1=\{0\} \times [0,b] \hspace{1ex} \Gamma_2=\{a\} \times [0,b] $

$u:\Omega \rightarrow \R^2 $ beschreibt die Verschiebung eines K�rpers auf dem Gebiet $\Omega$, wenn ein Riss entsteht. Der K�rper ist an $\Gamma_1$ und $\Gamma_2$ befestigt, was die Randbedingung $u_0$ angibt.  
 $v: \Omega \rightarrow \R$ gibt an, wo und wie stark der K�rper gerissen ist. 1 bedeutet, dass der K�rper vollst�ndig gerissen ist und 0, das kein Riss vorhanden ist.  
 
 Analytische und numerische Grundlagen werden gebraucht, um die Verschiebung des Gebietes und den Riss zu finden. F�r die analytische Betrachtung nutze ich partielle Differentialgleichungen und Grundlagen der Optimierung. Da die L�sung des Problems analytisch nicht zu finden ist, diskretisiere ich das Problem mit dreieckig lineare Lagrange Elemente und implementiere es mittels semiglatter Newtonmethoden. Die Grundlagen dazu sind im zweiten Kapitel zu finden. Alle Themen sind sehr umfangreich und ich werde nur die wichtigsten Begriffe einf�hren k�nnen. Der Leser sollte schon Wissen �ber Finite Elemente, insbesondere die dreieckig linearen Lagrange Elemente mitbringen.
 %todo brauche ich die noch nach meiner Einf�hrung?
 
 Die Untersuchung des Problems folgt in Kapitel drei. Beim ersten Betrachten f�llt auf, dass man die Optimierung nach $u$ und $v$ trennen kann. Dieses wird im ersten Teil des dritten Kapitels erl�utert. Im zweiten Teil wird die Optimierung nach $u$ betrachtet. Zuerst wird die Existenz und Eindeutigkeit gesichert, um dann numerisch die L�sung mit dreieckig linearen Lagrange Elementen zu suchen. Die Optimierung nach $v$ im dritten Teil hat den selben Aufbau. Nur ist hier das Problem komplizierter und die numerische L�sung erfolgt mit der semiglatten Newton Methoden. 
 Im letzten Kapitel werte ich die numerischen Resultate aus und ziehe R�ckschl�sse f�r die Konvergenz der Methode aufgrund der Gitterweite.  
%todo sp�ter �berpr�fen, ob ich das wirklich tue, wenns geschribene ist

