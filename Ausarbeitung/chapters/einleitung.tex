% ====================================
% Einleitung
% ====================================

\chapter{Einleitung}
\label{chap:einleitung}

%todo Konvergenz S 38 in Script 
%englischer Titel: solution of a phase field model for crack formation using semismooth newton methods

%todo anderer Anfangssatz
Wir wollen herausfinden wie Risse entstehen bzw wie sie sich fortsetzen.

%todo was ist ein phasenfeld?
%todo referenzen einf�gen!
Betrachten wir zun�chst ein Beispiel. Ein Blatt Papier befindet sich in Ruheposition auf dem Gebiet $\Omega \subset \R^3 $. Nun wird es an zwei gegen�berliegenden Seiten eingespannt und an diesen Seiten wird Kraft ausge�bt. Was passiert? Zun�chst wird sich das Blatt Papier ein wenig dehnen, also deformieren. Die neue Position des Papiers beschreibt $\varphi(x)$. Also gilt $\varphi=u+id$ wobei $u: \Omega \rightarrow \R^3$ die Verschiebung des K�rpers ist. 

Das Ziel ist es ein $\varphi$ zu finden, bei dem die Verzerrung des Papiers minimal ist. Dazu betrachten wir den Abstand zwischen $x \in \Omega$ und $ x+h \in \Omega$ im aktuellen Zustand
\begin{align*}
	\|\varphi(x+h)- \varphi(x)\|^2 = \|\nabla \varphi \|^2 + o(\|h\|^2) = h^T \nabla \varphi^T \nabla \varphi h + o(\|h\|^2)
\end{align*} 
Damit haben wir die Matrix $C:=\nabla \varphi^T \nabla \varphi $ erhalten, die die lokale �nderung der L�ngen angibt und Cauchy-Grennscher Verzerrungstensor genannt wird. 
Die Verzerrung $E$ l�sst sich �ber $ \frac{1}{2} (C-I)$ berechnen. Da $u=\varphi - \text{id}$ gilt, k�nnen wir die Ableitung von u einsetzen und erhalten
\begin{align*}
	E_{ij}=\frac{1}{2} \left( \frac{\partial  u_i}{\partial x_j} +\frac{\partial  u_j}{\partial x_i}  \right) + \frac{1}{2} \sum\limits_{k} \frac{\partial  u_i}{\partial x_k} \frac{\partial  u_j}{\partial x_k} 
\end{align*}
Da im linearen die quadratischen Terme vernachl�ssigt werden k�nnen, erhalten wir 
\begin{align*}
	\frac{1}{2} \left( \frac{\partial  u_i}{\partial x_j} +\frac{\partial  u_j}{\partial x_i} \right) = \frac{\nabla u^T + \nabla u}{2}
\end{align*}

Irgendwann h�lt das Papier den auf es wirkenden Kr�ften nicht mehr stand und rei�t. Sobald dies geschieht, gilt die Formel f�r die Verzerrung nicht mehr. Also brauchen wir einen Term, der angibt, wo und wie stark der K�rper gerissen ist. Dazu definieren wir $v: \Omega \rightarrow \R$. $v$ ist 0, falls der K�rper vollst�ndig gerissen ist und 1, wenn kein Riss vorhanden ist.  Multiplizieren wir also $ \frac{1}{2} (\nabla u^T + \nabla u)$ mit  $v^2 + \epsilon_1$. Es ergibt sich 
\begin{align*}
	& \min\limits_u \int\limits_{\Omega} |\nabla u|^2 (v^2 + \epsilon_1) \\
	& u=u_0 \text{ auf } \Gamma_1 \cup \Gamma_2  
\end{align*}
Da wir $u$ nur im zweidimensionalen betrachten, k�nnen wir  $\frac{\nabla u + \nabla u^T}{2}^2$  zu $|\nabla u|^2$ vereinfachen. Der K�rper ist an $\Gamma_1$ und $\Gamma_2$ befestigt, was die Randbedingung $u_0$ angibt.  

Ideal w�re es, wenn der K�rper niemals rei�en w�rde. Da dies aber in der Anwendung nicht der Fall sein kann, m�chten wir einen m�glichst kleinen Riss erhalten. Die Oberfl�che, auf die der Riss auftritt sollte also minimal sein. Wir wollen also, dass $\int\limits_{\Omega} (1-v)^2$ minimal ist. Aber in der Realit�t stellt man fest, dass der Riss glatt ist. Um dies zu modellieren, addieren wir die Ableitung zum Quadrat mit $\epsilon_2^2$ multipliziert hinzu. Daraus ergibt sich eine Darstellung vom Riss
\begin{align*}
	\epsilon_2 |\nabla v|^2 + \frac{1}{\epsilon_2} (1-v)^2
\end{align*}
Zusammengefasst ist das gesamte Problem gegeben durch
\begin{align*}
	& \min\limits_{u \in  \h^2 , v \in \h } \int_{\Omega} \left( v^2 + \epsilon_1 \right) | \nabla u|^2 + \nu \left(\epsilon_2 |\nabla v|^2 + \frac{1}{\epsilon_2} \left( 1- v \right)^2 \right) dx \\
	& \text{s.d.} \hspace{1ex} 0 \le v \le v_0 \\\
	& u=u_0 \text{ auf } \Gamma_1 \cup \Gamma_2  
\end{align*}
dabei ist $\nu$ eine Konstante, $\epsilon_i>0 \hspace{1ex}\forall i \in \{1,2\}$ ein kleiner Parameter und $\Gamma_1, \Gamma_2 \subset \Omega$  der rechte bzw. linke Rand eines rechteckigen Gebietes $\Omega \subset \R^2$, d.h., 
\begin{align*}
	\Omega=[0,a] \times [0,b] \hspace{1ex} \Gamma_1=\{0\} \times [0,b] \hspace{1ex} \Gamma_2=\{a\} \times [0,b] 
\end{align*}

Zur L�sung des Problems werden analytische und numerische Grundlagen ben�tigt. F�r die analytische Betrachtung nutze ich partielle Differentialgleichungen und Optimierung. Da die L�sung des Problems analytisch nicht zu finden ist, diskretisiere ich das Problem mit dreieckig linearen Lagrange Elementen und implementiere es mittels semiglatter Newton Methoden. Die Grundlagen dazu sind im zweiten Kapitel zu finden. Da jedes der eben genannten Themen sehr umfangreich ist, wird es mir nur m�glich sein die wichtigsten Begriffe einzuf�hren.
 
Die Untersuchung des Problems folgt in Kapitel drei. Beim ersten Betrachten f�llt auf, dass die Optimierung nach $u$ und $v$ getrennt werden kann. Dieses wird im ersten Teil des dritten Kapitels erl�utert. Im zweiten Teil wird die Optimierung nach $u$ betrachtet. Zuerst wird die Existenz und Eindeutigkeit gesichert, um dann numerisch die L�sung mit dreieckig linearen Lagrange Elementen zu suchen. Die Optimierung nach $v$ im dritten Teil hat den gleichen Aufbau. Nur ist hier das Problem komplizierter, da die Ableitung nicht berechnet werden kann und die numerische L�sung erfolgt mit der semiglatten Newton Methode. 
Im letzten Kapitel werte ich die numerischen Resultate aus und ziehe R�ckschl�sse f�r die Konvergenz der Methode aufgrund der Gitterweite.  
%todo sp�ter �berpr�fen, ob ich das wirklich tue, wenns geschrieben ist

