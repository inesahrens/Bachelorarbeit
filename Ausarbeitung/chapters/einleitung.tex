% ====================================
% Einleitung
% ====================================

\chapter{Einleitung}
\label{chap:einleitung}

%todo Konvergenz S 38 in Script 
%englischer Titel: solution of a phase field model for crack formation using semismooth newton methods


Ein Material wird an zwei Seiten eingespannt und Kraft wird an diesen Seiten ausge�bt. Die daraus resultierende Verzerrung und der Riss im Material wird durch ein Optimierungsproblem modelliert. In dieser Arbeit befasse ich mich mit der L�sung dieses Optimierungsproblems.  

Betrachten wir zum besseren Verst�ndnis der Modellierung des Problems ein Beispiel. Ein Blatt Papier befindet sich in Ruheposition auf dem Gebiet $\Omega \subset \R^3 $. Nun wird es an zwei gegen�berliegenden Seiten eingespannt und an diesen Seiten wird gezogen. Zun�chst dehnt sich das Blatt Papier ein wenig. Diese Verschiebung des Papiers gibt $u:\Omega \rightarrow \R^2 $ an. Wenn gen�gend Kr�fte auf dem Papier wirken, rei�t es. Dieser Riss wird durch die Abbildung $v: \Omega \rightarrow \R$ modelliert. $v$ ist 0, falls das Papier vollst�ndig gerissen ist und 1, wenn kein Riss vorhanden ist.

Das folgende Minimierungsproblem beschreibt, wie der Riss im Papier und die Verschiebung des Gebietes modelliert werden. 
\begin{align*}
	& \min\limits_{u \in  \h^2 , v \in \h } \int_{\Omega} \left( v^2 + \epsilon_1 \right) | \nabla u|^2 + \nu \left(\epsilon_2 |\nabla v|^2 + \frac{1}{\epsilon_2} \left( 1- v \right)^2 \right) dx \\
	& \text{s.d.} \hspace{1ex} 0 \le v \le v_0 \\
	& u=u_0 \text{ auf } \Gamma_1 \cup \Gamma_2  
\end{align*}
Dabei ist $\nu$ eine Konstante, $\epsilon_i>0$ f�r alle $i \in \{1,2\}$ ein kleiner Parameter und $\Gamma_1, \Gamma_2 \subset \Omega$  der rechte bzw. linke Rand eines rechteckigen Gebietes $\Omega \subset \R^2$, d.h., 
\begin{align*}
	\Omega=[0,a] \times [0,b] \hspace{1ex} \Gamma_1=\{0\} \times [0,b] \hspace{1ex} \Gamma_2=\{a\} \times [0,b] .
\end{align*}
Die Erkl�rung der genauen Modellierung erfolgt im Kapitel Modellierung \ref{chap:modellierung}. 

Zur L�sung des Minimierungsproblems werden analytische und numerische Grundlagen ben�tigt. F�r die analytische Betrachtung werden partielle Differentialgleichungen und Optimierung genutzt. Da die L�sung des Problems analytisch nicht zu finden ist, diskretisiere ich das Problem mit dreieckig linearen Lagrange Elementen und implementiere es mittels semiglatter Newton Methoden. Die Grundlagen dazu sind im zweiten Kapitel zu finden. Da jedes der eben genannten Themen sehr umfangreich ist, wird es mir nur m�glich sein, die wichtigsten Begriffe einzuf�hren.
 
Die Untersuchung des Problems folgt in Kapitel drei. Beim ersten Betrachten f�llt auf, dass die Optimierung nicht gleichzeitig nach $u$ und $v$ ausgef�hrt werden muss. Man kann zwei getrennte Optimierungsprobleme betrachten. Dieses wird im ersten Teil des dritten Kapitels erl�utert. Im zweiten Teil wird die Optimierung nach $u$ betrachtet. Zuerst wird die Existenz und Eindeutigkeit gesichert, um dann numerisch die L�sung mit dreieckig linearen Lagrange Elementen zu suchen. Die Optimierung nach $v$ im dritten Teil hat den gleichen Aufbau. Nur ist hier das Problem komplizierter, da die Ableitung nicht berechnet werden kann und die numerische L�sung erfolgt mit der semiglatten Newton Methode. 
Im letzten Kapitel werte ich die numerischen Resultate aus und veranschauliche meine Ergebnisse anhand mehrerer Anwendungsbeispiele.
%todo sp�ter �berpr�fen, ob ich das wirklich tue, wenns geschrieben ist

