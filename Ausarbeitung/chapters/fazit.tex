% ====================================
% Zusammenfassung, Fazit und Ausblick
% ====================================

\chapter{Fazit und Ausblick}
\label{chap:fazit_ausblick}
Ideen f�r Fazit, Ausblick, Auswertung

\begin{itemize}
	\item an $u_0, v_0$ weniger Vorraussetzungen, dann eindeutige L�sung?
	\item bei semismoothness weniger vorr an $u \in L^{\infty}$
	\item 
\end{itemize}

Optimierung nach u



Jetzt k�nnte man noch untersuchen, ob auch mit weniger Voraussetzungen an $u_0 $ und $v_0$ das Problem eine eindeutige L�sung hat. Dieses werde ich jedoch im Rahmen der Bachelorarbeit nicht untersuchen k�nnen.  
%todo vll untersuchen

Am Ende der Arbeit steht in der Regel die Zusammenfassung der geleisteten Arbeit. Hier werden noch einmal die zentralen Punkte der Aufgabenstellung und der Arbeit aufgelistet, ohne dabei die S�tze und Aussagen aus den vorherigen Teilen Arbeit w�rtlich zu wiederholen. Dieser Teil sollte auch andere Standpunkte der Arbeit aufzeigen, wie z.B.

\begin{itemize}
	\item Diskussion �ber Einschr�nkungen oder Begrenzungen der Arbeit oder der Methoden, ohne die Sachen dabei negativ zu formulieren.
	\item Darstellung von offenen Problemen, Perspektiven und Ausblicke f�r zuk�nftige Arbeiten.
\end{itemize}