% ====================================
% Zusammenfassung, Fazit und Ausblick
% ====================================

\chapter{Fazit und Ausblick}

Insgesamt l�sst sich feststellen, dass die Implementation bei einer gr��eren Anzahl an Gitterpunkten bei allen vorgestellten Beispielen realit�tsnahe Ergebnisse liefert. Bei weniger Gitterpunkten erhalte ich nicht das erwartete Ergebnis, wie z.B. bei Abbildung \ref{fig:riss_ganz_mitte_10_gitter.gebiet} und \ref{fig:riss_ganz_mitte_10_gitter.riss}. Hier rei�t das Material am Rand und nicht in der Mitte. Eine m�gliche Erkl�rung ist, dass der Riss sehr gro� im Vergleich zu dem Material ist und jegliche Fortsetzung des Risses schon dazu f�hrt, dass das Material auch am Rand gerissen wird. 

\begin{figure}[!bth]
	\centering	\includegraphics[scale=0.40]{images/optimazation/02_2_rissAnEinerSeite_100Gitter_u0Konstant.png}
	\caption{Darstellung des Risses  bei konstantem $u_0$, zwei kleinen Riss an einem Rand und $100x100$ Gitterpunkten}
	\label{fig:2riss_eineSeite_100_gitter.riss}
\end{figure}

Ein anderes, nicht funktionierendes Beispiel, zeigt die Modellierung von zwei Rissen, die in einem Material auf der gleichen Seite sind und aufeinander zu laufen. Ein Riss verschwindet komplett bei vielen Iterationen, wie in Abbildung \ref{fig:2riss_eineSeite_100_gitter.riss} dargestellt. Dies d�rfte nicht sein. Wenn das Material einmal gerissen ist, sollte der Riss immer dort bleiben. Weiterf�hrend k�nnte betrachtet werden, warum dieser Riss verschwindet. Vielleicht ist dies ein Problem in der Modellierung, da in dieser Modellierung die Oberfl�che des Risses klein gehalten wird, wenn die Verschiebung des Gebietes nicht zu gro� wird. Die Verschiebung des Materials wird klein, wenn schon ein anderer Riss vorhanden ist. Dadurch k�nnte der Riss verschwinden. 

Betrachten wir die Laufzeit der Implementation. Diese liegt bei $O(n^2)$, wobei $n$ die Breite bzw. H�he des Gitters ist. Dieses ist in \ref{fig:laufzeit} zu sehen. Eigentlich sollte es in $O(n)$ m�glich sein, das Problem zu implementieren, da nur Matrizen aufgestellt werden und mehrere Matrixmultiplikationen mit Sparse-Matrizen stattfinden, die sehr effizient sind. Beim genaueren Betrachten des Codes f�llt auf, dass meine Implementation sehr viel Zeit bei dem erstellen der Matrizen braucht. Dieses k�nnte vermutlich sehr viel effizienter berechnet werden. 

 \begin{figure}[!htb]
 	\centering	\includegraphics[scale=0.42]{images/laufzeit.png}
 	\caption{Laufzeit der Implemention. Auf der x-Achse ist die Gitterweite, auf der y-Achse die Zeit in Sekunden angegeben}
 	\label{fig:laufzeit}
 \end{figure}

In der analytischen Betrachtung k�nnte man untersuchen, ob weniger Voraussetzungen an $u_0$ und $v_0$ gestellt werden m�ssen. Stetigkeit von $u_0$ sollte ausreichend sein, ist aber auch notwendig. In der Praxis kann man Material nicht unstetig einspannen.  

Eine wichtige Untersuchung w�re die Konvergenz der Newton Methode. Diese zu beweisen erweist sich als sehr schwierig, da nicht klar ist, ob die Regularit�tsannahme erf�llt ist. Im Diskreten konvergiert die Methode f�r alle untersuchten Beispiele, was nat�rlich keine Aussage dar�ber trifft, ob sie auch im analytischen Sinn konvergiert. 

