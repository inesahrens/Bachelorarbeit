% ==================================
% Anhang - Allgemeine Informationen
% ==================================

\chapter{Allgemeine Informationen}
\label{chap:allg_info}

\section{Rechnungen}
\label{sec:rechnungen}
In diesem Kapitel werden alle Rechnungen vorgestellt, die zur numerischen Darstellung der Optimierung nach $v$ notwendig sind. 

\subsection{Numerische Darstellung von $G_1$}
Die Formel 
\begin{align*}
	G_1(v^h,\eta^h) & =\left(2 (\int\limits_{\Omega} u^{dis} T_i T_j \diff x)_{ij} + 2 \epsilon_2 (\int\limits_{\Omega} \nabla T_i \nabla T_j \diff x)_{ij} +  \frac{2}{\epsilon_3}  (\int\limits_{\Omega} T_i T_j \diff x)_{ij}  \right) v^k \\
	& - \frac{2}{\epsilon_3} (\int\limits_{\Omega} T_j \diff x)_j +  ( \int\limits_{\Omega} T_i T_j \diff x )_{ij} \eta^k \\
	& = (2 A + 2 \epsilon_2 B + \frac{2}{\epsilon_3} D) v^h  - \frac{2}{\epsilon_3} c + D \eta^h
\end{align*}
soll berechnet werden. 

Um $A,B,D$ zu berechnen, brauchen wir $\int_E \varphi_i \varphi_j$ bzw. $\int_E \nabla \varphi_i \nabla \varphi_j$, wobei E das Einheitsdreieck ist.
\begin{align*}
	\begin{array}{l|l|l}
		& \int_E \varphi_i \varphi_j & \int_E \nabla \varphi_i  \nabla \varphi_j \\
		\hline
		\varphi_0 \varphi_0  & \frac{1}{12} & 1 \\
		\varphi_1 \varphi_1  & \frac{1}{12} & \frac{1}{2} \\
		\varphi_2 \varphi_2  & \frac{1}{12} & \frac{1}{2} \\
		\varphi_0 \varphi_1  & \frac{1}{24} & -\frac{1}{2} \\
		\varphi_0 \varphi_2  & \frac{1}{24} & -\frac{1}{2} \\
		\varphi_1 \varphi_2  & \frac{1}{24} & 0   
	\end{array} 
\end{align*}

Nun k�nnen wir die einzelnen Matrizen berechnen.
Die Berechnung erfolgt analog zur Optimierung nach $u$. Bei den Matrizen gibt es immer die F�lle, dass $i$ und $j$ gleich sind, $j$ rechts neben $i$ ist, $j$ direkt unter $i$ liegt und $j$ rechts unter $i$ liegt. F�r alle anderen $i$ und $j$ ist der Matrixeintrag immer 0. Die Bezeichnungen sind die Gleichen, wie bei $u$. 
\paragraph{Berechnung der Matrix A}
\begin{align*}
	A_{i,i}	& = \int\limits_{\Omega}  u^{dis} T_i T_i \diff x\\
	& = \int\limits_{E_1} u^{dis}_{E_1} \varphi_1 \varphi_1 \diff x
	+ \int\limits_{E_2} u^{dis}_{E_2}  \varphi_2 \varphi_2 \diff x 
	+ \int\limits_{E_3} u^{dis}_{E_3} \varphi_0 \varphi_0 \diff x \\
	& \hspace{2ex}
	+ \int\limits_{E_4} u^{dis}_{E_4} \varphi_0 \varphi_0 \diff x
	+ \int\limits_{E_5} u^{dis}_{E_5} \varphi_2 \varphi_2 \diff x
	+ \int\limits_{E_6} u^{dis}_{E_6} \varphi_1 \varphi_1 \diff x \\
	& = \frac{1}{12} u^{dis}_{E_1} + \frac{1}{12} u^{dis}_{E_2} + \frac{1}{12}  u^{dis}_{E_3}+ \frac{1}{12}  u^{dis}_{E_4}+ \frac{1}{12} u^{dis}_{E_5}+ \frac{1}{12} u^{dis}_{E_6} \\
	& = \frac{1}{12} \left( \sum\limits_{i=1}^6 u^{dis}_{E_i} \right)
\end{align*}
%todo reihenfolge �ndern
\begin{align*}
	A_{i,i+1}	& = \int\limits_{\Omega}  u^{dis} T_i   T_{i+1} \diff x = \int\limits_{E_3}  u^{dis}_{E_3} \varphi_0 \varphi_1 \diff x
	+ \int\limits_{E_6} u^{dis}_{E_6}  \varphi_0 \varphi_1 \diff x \\
	& =  \frac{1}{24} u^{dis}_{E_3} + \frac{1}{24} u^{dis}_{E_6}  = \frac{1}{24} \left( u^{dis}_{E_3} + u^{dis}_{E_6} \right) 
\end{align*}
\begin{align*}
	A_{i,i+1+n}	& = \int\limits_{\Omega} u^{dis} T_i   T_{i+1+n} \diff x = \int\limits_{E_4} u^{dis}_{E_4} \varphi_0 \varphi_1 \diff x
	+ \int\limits_{E_5} u^{dis}_{E_5} \varphi_0 \varphi_1 \diff x \\
	& = \frac{1}{24} u^{dis}_{E_4} + \frac{1}{24} u^{dis}_{E_5} = \frac{1}{24} \left( u^{dis}_{E_4} + u^{dis}_{E_5} \right)
\end{align*}	
\begin{align*}
	A_{i,i+n+2}	& = \int\limits_{\Omega} u^{dis} T_i   T_{i+2+n} \diff x= \int\limits_{E_5} u^{dis}_{E_5} \varphi_1 \varphi_2 \diff x
	+ \int\limits_{E_6} u^{dis}_{E_6} \varphi_1 \varphi_2 \diff x \\
	& = \frac{1}{24} u^{dis}_{E_5} + \frac{1}{24} u^{dis}_{E_6}  = \frac{1}{24} \left( u^{dis}_{E_5} +u^{dis}_{E_6} \right)
\end{align*}


\paragraph{Berechnung der Matrix B}
\begin{align*}
	B_{i,i}	& = \int\limits_{\Omega} \nabla T_i \nabla T_i \diff x\\
	& = \int\limits_{E_1} \nabla \varphi_1 \nabla \varphi_1 \diff x
	+ \int\limits_{E_2} \nabla \varphi_2 \nabla \varphi_2 \diff x 
	+ \int\limits_{E_3} \nabla \varphi_0 \nabla \varphi_0 \diff x \\
	& + \int\limits_{E_4} \nabla \varphi_0 \nabla \varphi_0 \diff x
	+ \int\limits_{E_5} \nabla \varphi_2 \nabla \varphi_2 \diff x
	+ \int\limits_{E_6} \nabla \varphi_1 \nabla \varphi_1 \diff x \\
	& = \frac{1}{2} + \frac{1}{2} + 1 + 1 + \frac{1}{2} + \frac{1}{2}  = 4
\end{align*}
\begin{align*}
	B_{i,i+1}	& = \int\limits_{\Omega} \nabla T_i \nabla T_{i+1} \diff x= \int\limits_{E_3} \nabla \varphi_0 \nabla \varphi_1 \diff x
	+ \int\limits_{E_6} \nabla \varphi_0 \nabla \varphi_1 \diff x \\
	& = - \frac{1}{2} - \frac{1}{2} = -1 
\end{align*}
\begin{align*}
	B_{i,i+n+1}	& = \int\limits_{\Omega} \nabla T_i \nabla T_{i+1+n} \diff x= \int\limits_{E_4} \nabla \varphi_0  \nabla \varphi_1 \diff x
	+ \int\limits_{E_5} \nabla \varphi_0 \nabla \varphi_1 \diff x \\
	& = - \frac{1}{2} - \frac{1}{2}  = -1 
\end{align*}	  
\begin{align*}
	B_{i,i+n+2}	& = \int\limits_{\Omega} \nabla T_i \nabla T_{i+2+n} \diff x\\
	& = \int\limits_{E_5} \nabla \varphi_1  \nabla \varphi_2 \diff x
	+ \int\limits_{E_6} \nabla \varphi_1 \nabla \varphi_2 \diff x  = 0 
\end{align*}	

\paragraph{Berechnung des Vektors c}
\begin{align*}
	c:=\int\limits_{\Omega} T_i \diff x = \sum\limits_{E \in E_k} \int\limits_E T_i \diff x
\end{align*}
Wie immer reicht es, die sechs Dreiecke um den Gitterpunkt $i$ zu betrachten. Es gilt: 
\begin{align*}
	\int\limits_E T_i \diff x = \frac{1}{6} \hspace{1ex} \forall i
\end{align*}

Also berechnen wir 
\begin{align*}
	\int\limits_{\Omega} T_i \diff x & = \sum\limits_{E \in E_k} \int\limits_E T_i \diff x = \\
	& = \int\limits_{E_1} T_i \diff x+ \int\limits_{E_2} T_i \diff x+ \int\limits_{E_3} T_i \diff x+ \int\limits_{E_4} T_i\diff x + \int\limits_{E_5} T_i \diff x + \int\limits_{E_6} T_6 \diff x \\
	& = \frac{1}{6} + \frac{1}{6} + \frac{1}{6} + \frac{1}{6} + \frac{1}{6} + \frac{1}{6} = 1
\end{align*}

Falls $i$ an einem Rand liegen sollte, werden die Dreiecke, die nicht vorhanden sind, weggelassen. 

\paragraph{Berechnung der Matrix D}
\begin{align*}
	D_{i,i}	& = \int\limits_{\Omega}  T_i T_i \diff x\\
	& = \int\limits_{E_1}   \varphi_1 \varphi_1 \diff x
	+ \int\limits_{E_2}   \varphi_2 \varphi_2 \diff x
	+ \int\limits_{E_3}   \varphi_0 \varphi_0 \diff x \\
	& \hspace{2ex}
	+  \int\limits_{E_4}   \varphi_0 \varphi_0 \diff x
	+ \int\limits_{E_5}   \varphi_2 \varphi_2 \diff x
	+ \int\limits_{E_6}   \varphi_1 \varphi_1 \diff x\\
	& = \frac{1}{12} + \frac{1}{12} + \frac{1}{12} + \frac{1}{12} + \frac{1}{12} + \frac{1}{12}  = \frac{1}{2}
\end{align*}
\begin{align*}
	D_{i,i+1}	& = \int\limits_{\Omega}   T_i   T_{i+1} \diff x= \int\limits_{E_3}   \varphi_0 \varphi_1 \diff x
	+ \int\limits_{E_6}   \varphi_0 \varphi_1 \diff x \\
	& =  \frac{1}{24} + \frac{1}{24}  = \frac{1}{12}
\end{align*}
\begin{align*}
	D_{i,i+n+1}	& = \int\limits_{\Omega}   T_i   T_{i+1+n} \diff x = \int\limits_{E_4}   \varphi_0 \varphi_1 \diff x
	+ \int\limits_{E_5}   \varphi_0 \varphi_1 \diff x \\
	& = \frac{1}{24} + \frac{1}{24} = \frac{1}{12}
\end{align*}	 
\begin{align*}
	D_{i,i+n+2}	& = \int\limits_{\Omega}   T_i   T_{i+2+n} \diff x = \int\limits_{E_5}   \varphi_1 \varphi_2 \diff x
	+ \int\limits_{E_6}   \varphi_1 \varphi_2 \diff x \\
	& = \frac{1}{24} + \frac{1}{24}  = \frac{1}{12}
\end{align*}

\subsection{Berechnung von $G_{2v}$}


Es muss $\int\limits_{\Omega}  \frac{\partial f}{\partial v} T_i T_j \diff x $ berechnet werden. Dazu integriert man statt �ber $\Omega$ wieder �ber die einzelnen Dreiecke. 
Dabei ist zu beachten, dass f�r gerade und ungerade Dreiecke andere Ergebnisse zustande kommen:

\begin{align*}
	\begin{array}{ll|l|l}
		i&j & \int\limits_E  \frac{\partial f}{\partial v} T_i T_j \diff x \text{ gerades Dreieck} & \int\limits_E  \frac{\partial f}{\partial v} T_i T_j \diff x \text{ ungerades Dreieck} \\
		\hline
		0&0 	& \frac{1}{60} (f_1^h + 3 f_2^h + f_3^h) & \frac{1}{60} (f_1^h + 3 f_2^h + f_3^h) \\
		0&1	& \frac{1}{120} (f_1^h + 2f_2^h + 2f_3^h) & \frac{1}{120} (2f_1^h + 2f_2^h + f_3^h) \\
		0&2	& \frac{1}{120} (2f_1^h + 2f_2^h + f_3^h) & \frac{1}{120} (f_1^h + 2f_2^h + 2f_3^h) \\				
		1&1	& \frac{1}{60} (f_1^h + f_2^h + 3f_3^h) & \frac{1}{60} (3f_1^h + f_2^h + f_3^h) \\
		1&2	& \frac{1}{120} (2f_1^h + f_2^h + 2f_3^h) & \frac{1}{120} (2f_1^h + f_2^h +2 f_3^h) 	\\	
		2&2	& \frac{1}{60} (3f_1^h + f_2^h + f_3^h) & \frac{1}{60} (f_1^h + f_2^h + 3f_3^h) 
	\end{array}
\end{align*}
Dabei ist $f_1^h$ bei einem geraden Dreieck die Auswertung von $f^h$ an der oberen linken Ecke des Dreiecks. Die anderen Bezeichnungen sind darauf aufbauend. 

Damit k�nnen wir $\partial G_{2 v}$ diskretisieren. Wir nennen die Diskretisierung $F_{ij}$. Hier hat man wieder die vier F�lle: 

\begin{align*}
	F_{i,i}	& = \int\limits_{\Omega}  f^h T_i T_i \diff x\\
	& =  \int\limits_{E_1} f^h_{E_1} \varphi_1 \varphi_1 \diff x
	+ \int\limits_{E_2} f^h_{E_2}  \varphi_2 \varphi_2 \diff x
	+ \int\limits_{E_3} f^h_{E_3} \varphi_0 \varphi_0 \diff x \\
	& \hspace{2ex}
	+ \int\limits_{E_4} f^h_{E_4} \varphi_0 \varphi_0 \diff x
	+ \int\limits_{E_5} f^h_{E_5} \varphi_2 \varphi_2 \diff x
	+ \int\limits_{E_6} f^h_{E_6} \varphi_1 \varphi_1 \diff x\\
	& = \frac{1}{60} \left(  \left( f_1^h +  f_2^h + 3f_3^h\right)_{E_1} 
	+ \left( f_1^h +  f_2^h + 3f_3^h\right)_{E_2}
	+ \left( f_1^h +  3f_2^h + f_3^h\right)_{E_3} \right. \\
	& \left.
	+ \left( f_1^h +  3f_2^h + f_3^h\right)_{E_4} 
	+ \left( 3f_1^h +  f_2^h + f_3^h\right)_{E_5} 
	+ \left( 3f_1^h +  f_2^h + f_3^h\right)_{E_6} \right)
\end{align*}
\begin{align*}
	A_{i,i+1}	& = \int\limits_{\Omega}  f^h T_i   T_{i+1} \diff x= \int\limits_{E_3}  f^h_{E_3} \varphi_0 \varphi_1 \diff x
	+ \int\limits_{E_6} f^h_{E_6}  \varphi_0 \varphi_1 \diff x \\
	& =   \frac{1}{120} \left(  \left(  f_1^h +  2f_2^h + 2f_3^h\right)_{E_3} 
	+ \left( 2f_1^h +  2f_2^h + f_3^h\right)_{E_6} \right)
\end{align*}
\begin{align*}
	A_{i,i+1+n}	& = \int\limits_{\Omega} f^h T_i   T_{i+1+n} \diff x= \int\limits_{E_4} f^h_{E_4} \varphi_0 \varphi_1 \diff x
	+ \int\limits_{E_5} f^h_{E_5} \varphi_0 \varphi_1 \diff x \\
	& =   \frac{1}{120} \left(  \left( 2f_1^h +  2f_2^h + f_3^h\right)_{E_4} 
	+ \left( f_1^h +  2f_2^h + 2f_3^h\right)_{E_5}\right)
\end{align*}	
\begin{align*}
	A_{i,i+n+2}	& = \int\limits_{\Omega} f^h T_i   T_{i+2+n} \diff x= \int\limits_{E_5} f^h_{E_5} \varphi_1 \varphi_2 \diff x
	+ \int\limits_{E_6} f^h_{E_6} \varphi_1 \varphi_2 \diff x \\
	& =   \frac{1}{120} \left(  \left( 2 f_1^h +  f_2^h +2 f_3^h\right)_{E_5} 
	+ \left(2 f_1^h +  f_2^h + 2f_3^h\right)_{E_6}\right)
\end{align*}
hierbei bedeutet $\left( f_1^h +  f_2^h + f_3^h\right)_{E_j} $, dass $f_i^h$ $f$ auf dem $i$-ten Gitterpunkt des Dreieck $E_j$ ausgewertet wird. 
Die Transformation mit $1/h_1 h_2$ wird auch hier am Schluss ausgef�hrt. 

\section{Code}

first test
\lstinputlisting{code/calculate_u.m}