% ===============================================
% Anhang - Einbinden von Graphiken in der Arbeit
% ===============================================

% --- \chapter[Kurztitel]{Titel}
%			erzeugt eine formatierte und entsprechend der hierarchischen Ebene nummerierte Ueberschrift 'Titel'. Hier wird aber der 'Kurztitel' in der .toc-Datei gespeichert, um im Inhaltsverzeichnis verwendet zu werden. Au�erdem steht der 'Kurztitel' als Kolumnenbeschriftung zur Verfuegung.
\chapter[Einbinden von Graphiken]{Einbinden von Graphiken in der Arbeit}
\label{chap:graphiken}

Die richtige Einbindung von Graphiken bzw. numerischen Ergebnissen ist in der Bildverarbeitung ein besonders wichtiger Aspekt. Das Problem dabei ist, dass bei der Verwendung von Bildformaten mit verlustbehafteter Komprimierung m�glicherweise Artefakte in den Bildern auftreten k�nnten. Dies ist insbesondere unerw�nscht, wenn es auf die Genauigkeit oder auf den Vergleich der Ergebnisse ankommt. Deshalb sollten die Bilder f�r die numerischen Ergebnisse am besten in EPS oder PNG Format verwendet werden, da diese keine bzw. verlustfreie Komprimierung von Bildern erm�glichen. Sonstige Graphiken k�nnen aber auch in JPG Format eingebunden werden.

Die Einbindung externer Graphiken in LaTeX ist abh�ngig von der Art des Ausgabedokumentes und man sollte deshalb folgende Punkte beachten.

\begin{itemize}
	\item Typische Ausgabeformate DVI, PostScript und PDF k�nnen nur mit bestimmten Graphiktypen umgehen.
	\item Die Verwendung von pdfLaTeX, also die direkte Erzeugung von PDF Dokumenten, ist nur mit den Dateitypen PDF, JPG oder PNG zul�ssig. Die Vektorgraphiken im EPS Format sind bei pdfLaTeX nicht direkt verwendbar und m�ssen zun�chst in einen zul�ssigen Typ konvertiert werden, z.B. in eine PDF Datei mit Hilfe von GSview und Convert Option oder FreePDF.
	\item Eine M�glichkeit EPS Graphiken direkt einzubinden, ist die �bersetzung von LaTeX zu einer PDF Datei, die aus einer PostScript Dokument gewonnen wird. Im Fall von TeXnicCenter muss dazu nur das vordefinierte Ausgabeprofil \enquote{LaTeX \ $\Rightarrow$ \ PS \ $\Rightarrow$ \ PDF} ausgew�hlt werden. Beachte aber, dass diese Vorgehensweise keine Verwendung von PDF, PNG und JPG Dateien zul�sst.
\end{itemize}

Zum Schluss noch einige Hinweise und Bemerkungen zum Einbinden von Graphiken.

\begin{itemize}
	\item LaTeX ben�tigt f�r das Einbinden von Graphiken das Paket \textit{graphicx}.
	\item Gute Einf�hrungen zum Einbinden von Graphiken unter LaTeX findet man z.B. unter\\
		\url{http://downloads.chaosmos.de/doc/LaTeX/latex_etti_pub.pdf}\\
		\url{http://dante.ctan.org/tex-archive/macros/latex/required/graphics/grfguide.pdf}\\
		\url{ftp://ftp.tex.ac.uk/tex-archive/info/epslatex.pdf}
	\item Matlab erm�glicht einen direkten Export der Bilder in EPS, PNG oder JPG Format.
	\item Wird ein Bild nicht von einem selbst erstellt, sondern von einem anderen Autor �bernommen, dann muss hinter der Bildunterschrift eine Quellenangabe erscheinen.
\end{itemize}
