\documentclass[10pt,a4paper]{article}
\usepackage[ngerman]{babel}
\usepackage{amsmath}
\usepackage{amssymb}
\usepackage[utf8]{inputenc}

\begin{document}

\paragraph{1. Frage: Anwendung und Bedeutung des Problems}
(siehe teilweise mail)
Mir ist aufgefallen, dass ich mir keine Notizen zur Interpretation der einzelnen Terme des Optimierungsproblems gemacht habe. Können Sie mir noch mal erklären, was die einzelnen Terme bedeuten?

Ich wollte auch noch kurz was zur Anwendung von dem Optimierungsproblem schreiben. Wo braucht man Rissentstehung?

\paragraph{2. Frage: Zitieren}
Wie Zitiere ich richtig eine Internetquelle? die eine für die Darstellung von min + max ist nicht mehr online verfügbar...

Wie zitiere ich aus einer Vorlesung? ich habe bei der Theorie häufig was aus pdgl, der optimierungs II vorlesung und der numerik PDGL Vl sachen genommen.

\paragraph{3. Frage: stetigkeit von $ \partial G_1$}
Damit ich die semidifferenzierbarkeit von $G_1$ zeigen kann, muss $\partial G_1$ stetig Fréchet Differenzierbar sein. Fréchedifferenzierbarkeit bekomme ich hin. Stetigkeit ist noch ein Problem. Ist die Funktion
\begin{align*}
	& \partial G_{1 v} (v, \eta)(\phi) = \int\limits_{\Omega} 2 \varphi  \phi | \nabla u|^2 + \epsilon_2 2 \nabla \phi \nabla \varphi  + \frac{2}{\epsilon_3} \phi \varphi dx
\end{align*}
stetig in $\phi \in H^1$? Stetigkeit für alle Terme außer der $\nabla \phi \nabla \varphi $ Teil ist kein Problem mit $\epsilon -\delta$ kriterium. Wie kann ich den  $\nabla \phi \nabla \varphi $ Teil abschätzen? 

\paragraph{4. Frage: Konvergenz meiner Newton Methode}
Um die Konvergenz der Newton Methode zu zeigen, muss ich die Regularitätsannahme zeigen. Diese ist: 

	Betrachte  $G=0$ mit der Lösung $\overline{x}$. Dann lautet die Regularitätsannahme
	\begin{align}
	\label{eq:regularitaetsbedingung}
	\exists C>0, \quad \exists \delta >0 : \|M^{-1}\|_{X \rightarrow Y} \le C \quad \forall M \in \partial G(x) \quad \forall x \in X, \quad \|x-\overline{x}\|_X<\delta
	\end{align}


Bei mir gilt:
\begin{align*}
	M:= \begin{pmatrix}
	\partial G_{1 v} & \partial G_{1 \eta} \\
	\partial G_{2 v} & \partial G_{2 \eta} 
	\end{pmatrix}
\end{align*}
mit
\begin{align*}
	 \partial G_{1 v} (v, \eta)(\phi) &= \int\limits_{\Omega} 2 \varphi  \phi | \nabla u|^2 + \epsilon_2 2 \nabla \phi \nabla \varphi  + \frac{2}{\epsilon_3} \phi \varphi d x \\
	\partial G_{1 \eta} (v, \eta) (\phi) &= \int\limits_{\Omega}  \phi \varphi  dx \\
	\partial G_{2 v} (v, \eta)(\phi) & = \int\limits_{\Omega} \frac{\partial f}{\partial v} \varphi \phi dx  \\
	\partial G_{2 \eta} (v, \eta)(\phi) & = \int\limits_{\Omega} \frac{\partial f}{\partial \eta} \varphi \phi dx 	
\end{align*}
und
\begin{align*}
	\frac{\partial f}{\partial \eta} = 
	\left\{
	\begin{array}{ll}
	\{0\}		& \text{ falls }  -c(v-v_0) < \eta \text{ oder }  \eta < -cv \\
	\{1\} 		& \text{ falls } -cv < \eta < -c(v-v_0) \\
	\lbrack 0,1 \rbrack	& \text{ falls }  -c(v-v_0) = \eta \text{ oder }  \eta = -cv 
	\end{array}
	\right .
\end{align*}	
\begin{align*}
	\frac{\partial f}{\partial v}= 
	\left\{
	\begin{array}{ll}
	\{-c\}		& \text{ falls }  -c(v-v_0) < \eta \text{ oder }  \eta < -cv \\
	\{0\}		& \text{ falls } -cv < \eta < -c(v-v_0) \\
	\lbrack -c,0 \rbrack	& \text{ falls }  -c(v-v_0) = \eta \text{ oder }  \eta = -cv 
	\end{array}
	\right .
\end{align*}	
 
Ich habe Probleme beim Bilden der Inverse. Eigentlich sollte die Inverse ja sein:
\begin{align*}
	\frac{1}{\partial G_{1v} \partial G_{2 \eta} - \partial G_{2v} \partial G_{1 \eta} } 
	\begin{pmatrix}
		\partial G_{2 \eta} & - \partial G_{1 \eta}\\
		-  \partial G_{2v} & \partial G_{1v} 
	\end{pmatrix}
\end{align*}
Dafür muss $\partial G_{1v} \partial G_{2 \eta} - \partial G_{2v} \partial G_{1 \eta} \neq 0$ gelten. Gilt das überhaupt? z.b. für die $\phi=0$ gilt es nicht. 

\paragraph{5. Frage: Probleme mit dem Code}
(siehe mail)
ich habe jetzt soweit alles implementiert, aber die Ergebnisse, die ich bekomme, sehen "komisch" aus. Ich habe, wie schon bei der Optimierung nach u wieder Probleme am Rand. Ich weiß leider nicht, woher die kommen. Alle tests laufen soweit.
Außerdem sieht mein v bzw mein eta merkwürdig, da ich erwarten würde, dass sich der riss fortsetzt. Aber es ergibt sich ein komisches Muster. sodass abwechseln ein riss vorhanden ist und dann wieder nicht. 
\end{document}